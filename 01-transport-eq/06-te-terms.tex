\documentclass[12pt]{article}
\usepackage[top=1in, bottom=1in, left=1in, right=1in]{geometry}

\usepackage{setspace}
\onehalfspacing

\usepackage{amssymb}
%% The amsthm package provides extended theorem environments
\usepackage{amsthm}
\usepackage{epsfig}
\usepackage{times}
\renewcommand{\ttdefault}{cmtt}
\usepackage{amsmath}
\usepackage{graphicx} % for graphics files

% Draw figures yourself
\usepackage{tikz} 

% writing elements
\usepackage{mhchem}

% The float package HAS to load before hyperref
\usepackage{float} % for psuedocode formatting
\usepackage{xspace}

% from Denovo Methods Manual
\usepackage{mathrsfs}
\usepackage[mathcal]{euscript}
\usepackage{color}
\usepackage{array}

\usepackage[pdftex]{hyperref}
\usepackage[parfill]{parskip}

% math syntax
\newcommand{\nth}{n\ensuremath{^{\text{th}}} }
\newcommand{\ve}[1]{\ensuremath{\mathbf{#1}}}
\newcommand{\Macro}{\ensuremath{\Sigma}}
\newcommand{\rvec}{\ensuremath{\vec{r}}}
\newcommand{\omvec}{\ensuremath{\hat{\Omega}}}
\newcommand{\vOmega}{\ensuremath{\hat{\Omega}}}
%---------------------------------------------------------------------------
%---------------------------------------------------------------------------
\begin{document}
\begin{center}
{\bf NE 155/255, Fall 2019 \\
Assumptions and Terms of the Transport Equation\\
September 13, 2019}
\end{center}

\setlength{\unitlength}{1in}
\begin{picture}(6,.1) 
\put(0,0) {\line(1,0){6.25}}         
\end{picture}

%-----------------------------------------
\subsection*{Assumptions}

\begin{enumerate}
\item Particles are point objects ($\lambda = h/(mv)$) is small compared to 
      the atomic diameter): its state is fully described by its location, 
      velocity vector, and a given time. This ignores rotation and quantum 
      effects.
\item Neutral particles travel in straight lines between collisions.
\item Particle-particle collisions are negligible (makes TE linear).
\item Material properties are isotropic (generally valid unless velocities are 
      very low).
\item Material composition is time-independent (generally valid over short 
      time scales).
\item Quantities are expected values: fluctuations about the mean for very low 
      densities are not accounted for.
\end{enumerate}

\subsection*{The Transport Equation}

We consider a six-dimensional volume (as a six-dimensional cube) fixed in
space, of dimensions $\triangle x$, $\triangle y$, $\triangle z$,
$\triangle E$, $\triangle \theta$, $\triangle \varphi$. Then, the number of
particles within this volume at time $t$ is

\begin{equation*}
n(\rvec,E,\omvec,t)
\triangle x\triangle y\triangle z\triangle E\triangle \theta\triangle \varphi =
n(\rvec,E,\omvec,t)\triangle \beta,
\end{equation*}

where all arguments of $N$ are ``average'' arguments in the increment of
six-dimensional phase space $\triangle \beta$. The number of particles in this
cube changes with time:

\begin{equation*}
\triangle \beta\frac{\partial}{\partial t}n(\rvec,E,\omvec,t) =
\begin{array}{l}
\text{time rate of change of the number of}\vspace{-0.1cm}\\
\text{particles in the six-dimensional cube $\triangle \beta$.}
\end{array}
\end{equation*}

This time rate of change is due to five separate processes. One is the rate of
streaming of particles out of the volume through the boundaries. The other
processes occur within the six-dimensional ``cube'': the rate of absorption;
the rate of scattering from $E$, $\omvec$ to all other energies and directions,
known as outscattering; the rate of scattering into $E$, $\omvec$ from all
other energies and directions, known as inscattering; and the rate of
production of particles due to an internal source.

Now, let us consider the surfaces of the cube perpendicular to the $x$-axis.
For the net rate of particles leaving the cube through these two surfaces, we
have

\begin{equation*}
(\textrm{streaming})_x = \dot x n(\rvec,E,\omvec,t)\mid_x^{x+\triangle x}
\triangle y\triangle z\triangle E\triangle \theta \triangle\varphi,
\end{equation*}

where $\dot x$ is the $x$ component of the particle velocity, and 
$\triangle y\triangle z\triangle E\triangle\theta\triangle \varphi$ is the
surface area. Letting $\triangle x$ go to the differential $dx$, we rewrite

\begin{equation*}
(\textrm{streaming})_x = \triangle\beta \frac{\partial}{\partial x}
\left[\dot x n(\rvec,E,\omvec,t)\right].
\end{equation*}

Using the same procedure for the flow from the cube in the other five
``directions", we obtain

\begin{align*}
\textrm{streaming} =
\bigg[ \frac{\partial}{\partial x}(\dot x n) + &
\frac{\partial}{\partial y}(\dot y n) +\frac{\partial}{\partial z}(\dot z n) \\
&\quad\quad + \frac{\partial}{\partial E}(\dot E n) + 
\frac{\partial}{\partial \theta}(\dot \theta n) +
\frac{\partial}{\partial \varphi}(\dot \varphi n)\bigg] \triangle \beta,
\end{align*}

where $n = n(\rvec,E,\omvec,t)$.

The rate of absorption within the cube is the product of the number of
particles in the cube and the probability of absorption per particle per unit
of time. This probability is given by the product of the absorption cross
section and the particle speed $v$. That is,

\begin{equation*}
\textrm{absorption} = v\Sigma_a(\rvec,E)n(\rvec ,E,\omvec,t)\triangle \beta.
\end{equation*}

Using similar arguments and the fact that we need to sum the scattering from
(to) $E$, $\omvec$ to (from) all other energies and directions $E'$, $\omvec'$,
we find

\begin{align*}
\textrm{outscattering} &=
\triangle \beta \int_0^{\infty}\int_{4\pi}
v\Sigma_s(\rvec,E\rightarrow E', \omvec\rightarrow\omvec')
n(\rvec,E,\omvec,t)d\omvec'dE', \\
\textrm{inscattering} &= \triangle \beta \int_0^{\infty}\int_{4\pi}
v'\Sigma_s(\rvec,E'\rightarrow E, \omvec'\rightarrow\omvec)
n(\rvec,E',\omvec',t)d\omvec'dE',
\end{align*}

where $\Sigma_s(\rvec,E'\rightarrow E, \omvec'\rightarrow\omvec)$ is the
macroscopic differential scattering cross section. Since the distribution
function in the integrand of the outscattering term is independent of the
integration variables, we can rewrite outscattering as 
$\triangle \beta v\Sigma_s(\rvec,E)n(\rvec, E, \omvec,t)$.
Finally, we need to consider the internal source of particles. We quantify this
source by introducing the function $S(\rvec, E, \omvec, t)$ such that the rate
of introduction of particles into the cube is given by

\begin{equation*}
\textrm{source} = S(\rvec,E,\omvec,t)\triangle \beta.
\end{equation*}

\end{document}
