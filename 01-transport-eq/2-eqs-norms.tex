\documentclass[12pt]{article}
\usepackage[top=1in, bottom=1in, left=1in, right=1in]{geometry}

\usepackage{setspace}
\onehalfspacing

\usepackage{amssymb}
%% The amsthm package provides extended theorem environments
\usepackage{amsthm}
\usepackage{epsfig}
\usepackage{times}
\renewcommand{\ttdefault}{cmtt}
\usepackage{amsmath}
\usepackage{graphicx} % for graphics files

% Draw figures yourself
\usepackage{tikz} 

% The float package HAS to load before hyperref
\usepackage{float} % for psuedocode formatting
\usepackage{xspace}

% from Denovo Methods Manual
\usepackage{mathrsfs}
\usepackage[mathcal]{euscript}
\usepackage{color}
\usepackage{array}

\usepackage[pdftex]{hyperref}
\usepackage[parfill]{parskip}

% math syntax
\newcommand{\nth}{n\ensuremath{^{\text{th}}} }
\newcommand{\ve}[1]{\ensuremath{\mathbf{#1}}}
\newcommand{\Macro}{\ensuremath{\Sigma}}
\newcommand{\rvec}{\ensuremath{\vec{r}}}
\newcommand{\vecr}{\ensuremath{\vec{r}}}
\newcommand{\omvec}{\ensuremath{\hat{\Omega}}}
\newcommand{\vOmega}{\ensuremath{\hat{\Omega}}}
\newcommand{\sigs}{\ensuremath{\Sigma_s(\rvec,E'\rightarrow E,\omvec'\rightarrow\omvec)}}
\newcommand{\el}{\ensuremath{\ell}}
\newcommand{\sigso}{\ensuremath{\Sigma_{s,0}}}
\newcommand{\sigsi}{\ensuremath{\Sigma_{s,1}}}
\newcommand{\vecx}{\ensuremath{\vec{x}}}
\newcommand{\vecy}{\ensuremath{\vec{y}}}

\newtheorem{theorem}{Theorem}
\newenvironment{definition}[1][Definition]{\begin{trivlist}
\item[\hskip \labelsep {\bfseries #1}]}{\end{trivlist}}
\newenvironment{example}[1][Example]{\begin{trivlist}
\item[\hskip \labelsep {\bfseries #1}]}{\end{trivlist}}

%---------------------------------------------------------------------------
\begin{document}
\begin{center}
{\bf NE 155/255 \\
Equations and Vector Norms \\ September 04, 2019}
\end{center}

\setlength{\unitlength}{1in}
\begin{picture}(6,.1) 
\put(0,0) {\line(1,0){6.25}}         
\end{picture}

\section{Equations, continued}
%%-------------------------------------------------------------
%%-------------------------------------------------------------
\subsection*{Integro-Differential Equations}
%https://en.wikipedia.org/wiki/Integro-differential_equation

...are equations that involve both integrals and derivatives of a function.

The general first-order, linear (only with respect to the term involving the 
derivative) integro-differential equation is of the form

\[
\frac{d}{dx}u(x) + \int_{x_0}^{x} f(t,u(t)) dt = g(x,u(x))\:, \qquad u(x_0) =
u_0 \:, \qquad x_0 \geq 0\:.
\]

This is the equation type we will likely deal with the most.

\underline{Nuclear Engineering Example:}

One-dimensional in space, one-dimensional in angle, time-independent, 
monoenergetic neutron transport equation: 

\[
\mu \frac{\partial \psi(x,\mu)}{\partial x} + \Sigma_t \psi(x,\mu) =
\frac{\Sigma_s}{2}\int_{-1}^{1} d\mu' \psi(x,\mu') + S(x, \mu)
\]

where the angular neutron flux is a function of one spatial variable ($x$) and 
one angular variable ($\mu = \cos(\theta)$).

%%-------------------------------------------------------------
%%-------------------------------------------------------------
\subsection*{Integral Equations}
%https://en.wikipedia.org/wiki/Integral_equation

...are equations in which an unknown function appears under an integral sign.

Integral equations are classified according to three different dichotomies, 
creating eight different kinds:

\begin{enumerate}
\item Limits of integration
  \begin{enumerate}
  \item both fixed: Fredholm equation
    \[f(x)=\int _{a}^{b}K(x,t)\,\varphi (t)\,dt\]
  \item one variable: Volterra equation
    \[f(x)=\int _{a}^{x}K(x,t)\,\varphi (t)\,dt\]
  \end{enumerate}

\item Placement of unknown function
  \begin{enumerate}
  \item only inside integral: first kind (both above examples)
  \item both inside and outside integral: second kind
    \[\varphi (x)=f(x)+\lambda \int _{a}^{x}K(x,t)\,\varphi (t)\,dt\]
  \end{enumerate}

\item Nature of known function, $f$
  \begin{enumerate}
  \item identically zero: homogeneous
  \item not identically zero: inhomogeneous
  \end{enumerate}
\end{enumerate}

Both Fredholm and Volterra equations are linear integral equations, due to the 
linear behaviour of $\phi(x)$ under the integral. 
A nonlinear Volterra integral equation has the general form

\[ \varphi (x)=f(x)+\lambda \int _{a}^{x}K(x,t)\,F(x,t,\varphi (t))\,dt\]

where $F$ is a known function.

The integral form of the Neutron Transport Equation is

\[
\psi(\rvec, \vOmega, E) =
\int_0^{\infty} d\rho' \:\exp\left[-\int_0^{\rho'} d\rho'' \:
\Sigma_t(\rvec-\rho''\vOmega,E)\right]q(\rvec-\rho'\vOmega,\vOmega,E)\]

where $q$ contains fixed, inscattering, and fission sources. That is,
$q = f(\psi)$.

%------------------------------------------------------------------------------
\section{Vector Norms}

\begin{definition}

A \textit{Vector Norm} on $\mathbb{R}^n$ is a function $||\cdot||$ mapping
$\mathbb{R}^n \rightarrow \mathbb{R}$ with the following properties:

\begin{enumerate}
\item $|| \vecx|| \geq 0$ for all $\vecx \in \mathbb{R}^n$
\item $|| \vecx|| = 0$ iff $\vecx = 0$
\item $|| \alpha\vecx|| = |\alpha| ||\vecx||$ for all $\alpha\in\mathbb{R}$  
and $\vecx \in \mathbb{R}^n$ (scalar multiplication)
\item $|| \vecx + \vecy|| \leq ||\vecx|| + ||\vecy||$ for all $\vecx,\vecy \in 
\mathbb{R}^n$ (triangle inequality)
\end{enumerate}
\end{definition}
\pagebreak

\underline{\textbf{Common Norms}}

\begin{itemize}
\item The $l_1$ norm is given by
\[
||\vecx||_1 = \sum_{i=1}^n|x_i|
\]
\item The \textit{Max norm, Sup norm}, or $l_\infty$ norm, is given by
\[
||\vecx||_\infty = \max_{1\leq i\leq n} |x_i|
\]
\item The \textit{Euclidean Norm}, or $l_2$ norm, is given by
\[
||\vecx||_2 = \left(\sum_{i=1}^n x_i^2\right)^{1/2}
\]
Note: this norm represents the usual notion of distance (Pythagorean theorem)
\end{itemize}

\begin{example}
Consider $\vecx = (1,-3,5)^T$. Then
\begin{itemize}
\item[i.] $||\vecx||_1 = 9$
\item[ii.] $||\vecx||_\infty = 5$
\item[iii.] $||\vecx||_2 = \sqrt{1+9+25} = \sqrt{35}$
\end{itemize}
\end{example}

\begin{center}$-----------------$\end{center}

\vspace*{1em}
\underline{\textbf{Cauchy-Schwarz and Triangle Inequality}}

It is easy to see that all the properties of a vector norm are satisfied for
$l_1$ and $l_\infty$, but we need to show that the triangle inequality holds 
for $l_2$. We will need the following theorem:

\vspace*{1em}

\begin{theorem}
(Cauchy-Schwarz in $\mathbb{R}^n$)\\
For each $\vecx,\vecy \in \mathbb{R}^n$,
\[
\vecx^T\vecy =
\sum_{i=1}^n x_iy_i\leq\left(\sum_{i=1}^nx_i^2\right)^{1/2}
\left(\sum_{i=1}^ny_i^2\right)^{1/2} = ||\vecx||_2||\vecy||_2
\]
\end{theorem}
\begin{proof}
Consider the following quadratic polynomial in $z\in\mathbb{R}$:

\[
0\leq(x_1z+y_1)^2+...+(x_nz+y_n)^2 = \left(\sum_{i=1}^nx_i^2\right)z^2 +
2\left(\sum_{i=1}^nx_iy_i\right)z + \left(\sum_{i=1}^ny_i^2\right).
\]

Since it is nonnegative, it has at most one real root for $z$.
Hence, its discriminant is less than or equal to zero; that is,

\[
\left(\sum_{i=1}^nx_iy_i\right)^2-
\left(\sum_{i=1}^nx_i^2\right)\left(\sum_{i=1}^ny_i^2\right)\leq 0.
\]
\end{proof}

\vspace*{1em}
\underline{\textbf{Proof of the Triangle Inequality for $l_2$}}:
\begin{proof}
Using Cauchy-Schwarz, for each $\vecx,\vecy \in \mathbb{R}^n$:
\[
||\vecx + \vecy||_2^2 = \sum_{i=1}^n(x_i+y_i)^2 = \sum_{i=1}^nx_i^2 +
2\sum_{i=1}^nx_iy_i + \sum_{i=1}^ny_i^2 \leq ||\vecx||_2^2 +
2||\vecx||_2||\vecy||_2 + ||\vecy||_2^2.
\]
Taking the square root of both sides, we obtain
\[
||\vecx + \vecy||_2 \leq ||\vecx||_2+||\vecy||_2.
\]
\end{proof}

\end{document}