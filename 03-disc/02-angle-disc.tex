\documentclass[12pt]{article}
\usepackage[top=1in, bottom=1in, left=1in, right=1in]{geometry}

\usepackage{setspace}
\onehalfspacing

\usepackage{amssymb}
%% The amsthm package provides extended theorem environments
\usepackage{amsthm}
\usepackage{epsfig}
\usepackage{times}
\renewcommand{\ttdefault}{cmtt}
\usepackage{amsmath}
\usepackage{graphicx} % for graphics files
\usepackage{tabu}

% Draw figures yourself
\usepackage{tikz} 

% writing elements
%\usepackage{mhchem}

\usepackage{paralist}

% The float package HAS to load before hyperref
\usepackage{float} % for psuedocode formatting
\usepackage{xspace}

% from Denovo Methods Manual
\usepackage{mathrsfs}
\usepackage[mathcal]{euscript}
\usepackage{color}
\usepackage{array}

\usepackage[pdftex]{hyperref}
\usepackage[parfill]{parskip}

% math syntax
\newcommand{\nth}{n\ensuremath{^{\text{th}}} }
\newcommand{\ve}[1]{\ensuremath{\mathbf{#1}}}
\newcommand{\Macro}{\ensuremath{\Sigma}}
\newcommand{\rvec}{\ensuremath{\vec{r}}}
\newcommand{\vecr}{\ensuremath{\vec{r}}}
\newcommand{\omvec}{\ensuremath{\hat{\Omega}}}
\newcommand{\vOmega}{\ensuremath{\hat{\Omega}}}
\newcommand{\even}{\ensuremath{\phi^g}}
\newcommand{\odd}{\ensuremath{\vartheta^g}}
\newcommand{\evenp}{\ensuremath{\phi^{g'}}}
\newcommand{\oddp}{\ensuremath{\vartheta^{g'}}}
\newcommand{\Sn}{\ensuremath{S_N} }
\newcommand{\Ye}[2]{\ensuremath{Y^e_{#1}(\vOmega_#2)}}
\newcommand{\sigg}[1]{\ensuremath{\Macro^{g'\rightarrow g}_{s,#1}}}
\newcommand{\psig}{\ensuremath{\psi^g}}
%---------------------------------------------------------------------------
%---------------------------------------------------------------------------
\begin{document}
\begin{center}
{\bf NE 155/255, Fall 2019 \\
Angular Discretization\\
September 30, 2019}
\end{center}

\setlength{\unitlength}{1in}
\begin{picture}(6,.1) 
\put(0,0) {\line(1,0){6.25}}         
\end{picture}

%---------------------------------------------------------------------------
%---------------------------------------------------------------------------
\section*{Angular Discretization}

We have two main approaches to angular
disrectization, the \textbf{Discrete Ordinates} method and the 
\textbf{Spherical Harmonics} method (which we often simplify to 
Legendre polynomials and thus call $P_N$, not to be confused with the 
scattering expansion).

\subsection*{Discrete Ordinates}

The discrete ordinates approximation is a collocation method in angle. A 
collocation method is a solution method for ODEs, PDEs, and integral 
equations. Choose a finite-dimensional space of candidate solutions, such as 
polynomials up to a certain degree, and a number of points within the domain, 
called collocation points. Select the solution that satisfies the equation at 
those points within that space. 
%https://en.wikipedia.org/wiki/Collocation_method

For us, the collocation points are the discrete angles that we choose
($\omvec \rightarrow \omvec_a; a = 1,\dots,n)$ and the solution space is the 
flux. The TE is only valid along the selected set of angles $\vOmega_a$. We 
apply a compatible quadrature (integration approximation) to the integral 
term. We write one equation for each angle in the set (dropping energy 
dependence and fission for simplicity; the source term contains scattering and 
external sources):

\begin{align*}
\vOmega_a \cdot \nabla \psi_a(\vecr) &+ 
\Sigma_t(\vecr)\psi_a(\vecr) = Q_a(\vec{r})\\
\psi_a(\vecr) &\equiv \psi(\vecr,\vOmega_a) \qquad Q_a(\vecr) \equiv 
Q(\vecr,\vOmega_a)\\
\int_{4\pi} d\vOmega\: &= \sum_{a=1}^n w_a = 4\pi \\
\phi(\vecr) &=  \int_{4\pi} d\vOmega\:\psi(\vecr,\vOmega) = 
\sum_{a=1}^n w_a \psi_a(\vecr)
\end{align*}

The collection ($\vOmega_a, w_a$) is known as the angular quadrature set. The 
$w_a$ are the integration weights that go with the angles to create an 
integration. The angle-weight combination + number of angles dictates the 
accuracy of the integration. The quadrature we choose also dictates the number 
of unknowns. For level-symmetric quadratures, the most common types and what
people usually mean by $S_{N}$, we get $n=N(N+2)$ unknowns. 

However, we still need to explain what's going on in the sources. To do that, 
we're going to look at \textit{Spherical Harmonics} and how they relate to 
Legendre Polynomials. This will allow us to do angular expansions in three 
dimensions (derived from the Exnihilo manual and Wikipedia). 

\subsubsection*{About Spherical Harmonics}

The addition theorem of Spherical Harmonics can be used to evaluate the
Legendre function $P_{\ell}(\vOmega'\cdot\vOmega)$,
\begin{equation}
  P_{\ell}(\vOmega'\cdot\vOmega) = \frac{4\pi}{2\ell +1}\sum_{m=-\ell }^\ell 
  Y_{\ell m}(\vOmega)Y^{\ast}_{\ell m}(\vOmega')\:,
\end{equation}
where the spherical harmonics $Y_{\ell m}$ are
\begin{equation}
  Y_{\ell m}(\theta,\varphi) = (-1)^m\sqrt
  {
    \frac{2\ell +1}{4\pi}\frac{(\ell-m)!}{(\ell+m)!}
  }\,
  P_{\ell m}(\cos\theta)\mathrm{e}^{im\varphi}\:,
  \label{eq:complete-spherical-harmonics}
\end{equation}
and the $P_{\ell m}$ are the \textit{associated Legendre Polynomials}. These are 
the solutions to

\[
(1-x^{2}){\frac {d^{2}}{dx^{2}}}P_{\ell }^{m}(x)-
2x{\frac {d}{dx}}P_{\ell }^{m}(x)+\left[\ell (\ell +1)-
{\frac {m^{2}}{1-x^{2}}}\right]P_{\ell }^{m}(x)=0\:,
\]

where the indices $\ell$ and $m$ are referred to as the degree and order of the 
associated Legendre polynomial, respectively. When $m$ is zero, these functions
are identical to the Legendre polynomials.
%https://en.wikipedia.org/wiki/Associated_Legendre_polynomials

We're going to use Spherical Harmonics to expand our scattering and external 
source. Everything in our equations must be real; therefore, we can follow a 
methodology that shows expands a real-valued function using complex Spherical 
Harmonics ($Y^*$ indicates complex conjugate).  First,
the expansion is split into positive and negative components of $m$:

\begin{equation}
P_\ell(\vOmega'\cdot\vOmega) = \frac{4\pi}{2\ell+1}
\Bigl[
Y_{\ell0}(\vOmega)Y_{\ell0}(\vOmega') +
\sum_{m=1}^l
\bigl(Y_{\ell m}(\vOmega)Y^{\ast}_{\ell m}(\vOmega') +
Y_{\ell-m}(\vOmega)Y^{\ast}_{\ell -m}(\vOmega')\bigr)\Bigr]\:.
\end{equation}

Next, we're going to go through a bunch of mathematical maneuvers to get rid 
of the negative $m$ components and the complex conjugate values. The 
motivation is ease of analysis and handling.

First, we'll simplify the $m=0$ term:

\begin{equation}
Y_{\ell 0} = \sqrt{\frac{2\ell +1}{4\pi}}\,P_{\ell 0} = Y^e_{\ell 0}\:,
\end{equation}

where we have used the idea that we can separate a spherical harmonic into 
even and odd components using the $e^{ix}$ identity,

\begin{align}
Y^e_{\ell m} &= D_{\ell m}P_{\ell m}\cos (m\varphi)\:,\label{eq:Ye}\\
Y^o_{\ell m} &= D_{\ell m}P_{\ell m}\sin (m\varphi)\:,\label{eq:Yo}\\
D_{\ell m} &= (-1)^m\sqrt{(2-\delta_{m0})\frac{2\ell+1}{4\pi}\frac{(\ell-m)!}{(\ell+m)!}}\:.
\end{align}

The \(\sqrt{(2-\delta_{m0})}\) term appears as an orthogonality requirement. 
When we expand a function in an entirely \textit{real} basis (as opposed to 
the complex basis of the general spherical harmonics), the basis functions for 
the real basis must be orthogonal. The spherical harmonics basis functions are 
orthogonal, but they also contain complex components that we don't need for 
real physical quantities such as the scattering source. Hence, the basis 
functions for the real basis need to be orthogonal, which means that the 
integral of \(\hat{Y}^e_{\ell m}\hat{Y}^e_{\ell'm'}\) over the unit sphere must give 
unity (with some Kronecker Deltas hanging around as well). We likewise require 
this orthogonality condition for the odd components. These two orthogonality 
statements do not give values of 1 unless the definitions contain an extra 
factor of \(\sqrt{(2-\delta_{m0})}\) (the \(\delta_{m0}\) term appears from 
the orthogonality of cosines in the definition of \(\hat{Y}_{lm}^e\)). 

We will make this substitution in order to satisfy the orthogonality 
requirement:

\begin{equation}
\hat{Y}^e_{lm} = \frac{1}{\sqrt{2-\delta_{m0}}}Y^e_{lm}\:,\quad
\hat{Y}^o_{lm} = \frac{1}{\sqrt{2-\delta_{m0}}}Y^o_{lm}\:.
\end{equation}

Because \(\hat{Y}_{\ell0}^o=0\) based on the inclusion of \(\sin(m\varphi)\) in 
its definition, the above sometimes neglects the \(\delta_{m0}\) term in the 
odd function since it is implied that \(m=0\) leads to an identically zero 
function anyways.

\end{document}
