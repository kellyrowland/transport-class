\documentclass[12pt]{article}
\usepackage[top=1in, bottom=1in, left=1in, right=1in]{geometry}

\usepackage{setspace}
\onehalfspacing

\usepackage{amssymb}
%% The amsthm package provides extended theorem environments
\usepackage{amsthm}
\usepackage{epsfig}
\usepackage{times}
\renewcommand{\ttdefault}{cmtt}
\usepackage{amsmath}
\usepackage{graphicx} % for graphics files
\usepackage{tabu}

% Draw figures yourself
\usepackage{tikz} 

% writing elements
\usepackage{mhchem}

\usepackage{paralist}

% The float package HAS to load before hyperref
\usepackage{float} % for psuedocode formatting
\usepackage{xspace}

% from Denovo Methods Manual
\usepackage{mathrsfs}
\usepackage[mathcal]{euscript}
\usepackage{color}
\usepackage{array}

\usepackage[pdftex]{hyperref}
\usepackage[parfill]{parskip}

% math syntax
\newcommand{\nth}{n\ensuremath{^{\text{th}}} }
\newcommand{\ve}[1]{\ensuremath{\mathbf{#1}}}
\newcommand{\Macro}{\ensuremath{\Sigma}}
\newcommand{\rvec}{\ensuremath{\vec{r}}}
\newcommand{\omvec}{\ensuremath{\hat{\Omega}}}
\newcommand{\vOmega}{\ensuremath{\hat{\Omega}}}
%---------------------------------------------------------------------------
%---------------------------------------------------------------------------
\begin{document}
\begin{center}
{\bf NE 155/255, Fall 2019 \\
Solution Approaches\\
September 25, 2019}
\end{center}

\setlength{\unitlength}{1in}
\begin{picture}(6,.1) 
\put(0,0) {\line(1,0){6.25}}         
\end{picture}

Last class we talked about the reactor problem types and physics involved, as 
well as how that necessarily impacts the solution strategies we use. Today 
we'll talk about approaches and what kinds of methods we use when. 

%-------------------------------------------------------------
%-------------------------------------------------------------
\subsection*{Computational Methods}

Inside the box of computational methods, we have two main categories: 

\begin{compactitem}
\item \textbf{Deterministic}: discretize all independent variables, obtain a 
      set of coupled, linear, algebraic equations and develop a numerical 
      method to solve them.
\item \textbf{Probabilistic}: follow the history of each relevant particle 
      based on the underlying probabilities for various types of interactions.
\end{compactitem}

Everything we've done so far applies to both methods (we've only simplified 
the equation). Each method has its own benefits and challenges. We'll talk a 
bit about \textbf{deterministic} methods first. Here, the discretization methods are a 
big part of the strategy, followed by parallelization.

The size of the problems (which impacts memory and parallelization 
performance) is governed by discretization. The quality and behavior of 
solutions is also governed by discretization. To get more accurate fluxes, 
typical transport problems today are three-dimensional, have up to thousands
$\times$ thousands $\times$ thousands of mesh points, use up to $\sim$ 150 
energy groups, include accurate expansions of scattering terms, and are solved 
over many directions. 

\begin{table}[!h]
\caption{Meaning and Range of Indices Used in Transport Discretization}
\begin{center}
\begin{tabular}{l c c c c c c}
\hline
Variable & Symbol & First & Last & Low & High \\[0.5ex]
\hline
Energy & $g$ & 0 or 1 & $G-1$ or $G$ & 2 & 100s \\
Solid Angle & $a$ & 1 & $n$ & $S_2$ & $S_{16}$ \\
Space & $i,j,k$ & 0,0,0 & $I,J,K$ & 1? & 1e9 \\
Legendre moment ($P_{N}$) & $\ell$ & 0 & $N$ & 0 & 9 \\
%Spherical harmonic moment ($Y$) & m & 0 & $l$ \\
\hline
\end{tabular}
\end{center}
\label{table:index}
\end{table}

E.g.\ In 2012, a Pressurized Water Reactor (PWR)-900 with 44 groups, a 578
$\times$ 578 $\times$ 700 mesh, using $S_{16}$ level-symmetric quadrature, and 
with $P_{0}$ scattering was solved: 1.7 trillion unknowns.

With \textbf{Monte Carlo} methods, we sample the physics of every single interaction 
of every single particle until we have enough samples to assert that something 
is statistically valid. This requires lots and lots of samples, but is 
typically fairly easy to parallelize. Monte Carlo methods require strategies and 
rules for sampling the physics, which can be complicated. 

\begin{figure}[h!]
    \begin{center}
    \includegraphics[keepaspectratio, width = 4 in]{MC}
    \end{center}
    \caption{Monte Carlo particle tracking}
    \label{fig:mc}
\end{figure}

\subsection*{Comparison}

In order of increasing accuracy\textit{ and} increasing runtime:

\begin{enumerate}
\item Diffusion Theory
  \begin{compactitem}
  \item discretized and homogenized space
  \item linearly anisotropic direction
  \item discretized energy (few-group)
  \end{compactitem}
\item Deterministic
  \begin{compactitem}
  \item discretized space
  \item discretized direction (discrete ordinates [$S_N$]) or 
        functional expansion of direction (spherical harmonics)
  \item discretized energy (multi-group)
  \end{compactitem}
\item Monte Carlo
  \begin{compactitem}
  \item continuous spatial resolution
  \item continuous direction representation
  \item continuous energy representation
  \end{compactitem}
\end{enumerate}


\begin{center}
\begin{small}
\begin{tabu}{| l | X | X |}
  \hline
  & Monte Carlo         & Deterministic \\\hline
    % -----------------------
    Strengths & * General geometry    & * Fast \\
              & * Continuous energy   & * Global solution\\
              & * Continuous in engle & * Solution is of same quality 
                                          everywhere\\
              & * Inherently 3-D      & * Inputs can be relatively simple \\
              & * Easy to parallelize on CPUs & \\\hline
    % -----------------------
    Weaknesses & * Slow & * Discretization governs solution quality \\
               & * Might be memory intensive & * Might be memory intensive \\
               & * Solutions have statistical error & * Solution contains 
                                                        truncation error\\
               & * Local solutions only & * Constrained by what you can mesh \\
               & * Must adequately sample phase space & * Ray effects \\
               & * Need efficient variance reduction (VR) & * Can be complicated to parallelize \\
               & * Input can be extremely complicated & \\\hline
\end{tabu}
\end{small}
\end{center}
We tend to think of Monte Carlo methods as ``benchmark quality" and highly flexible. \\
What is one major hazard of Monte Carlo methods?\\
What is the biggest challenge of using MC methods for detailed design?

\end{document}