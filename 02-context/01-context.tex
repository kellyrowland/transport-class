\documentclass[12pt]{article}
\usepackage[top=1in, bottom=1in, left=1in, right=1in]{geometry}

\usepackage{setspace}
\onehalfspacing

\usepackage{amssymb}
%% The amsthm package provides extended theorem environments
\usepackage{amsthm}
\usepackage{epsfig}
\usepackage{times}
\renewcommand{\ttdefault}{cmtt}
\usepackage{amsmath}
\usepackage{graphicx} % for graphics files
\usepackage{tabu}

% Draw figures yourself
\usepackage{tikz} 

% writing elements
\usepackage{mhchem}

\usepackage{paralist}

% The float package HAS to load before hyperref
\usepackage{float} % for psuedocode formatting
\usepackage{xspace}

% from Denovo Methods Manual
\usepackage{mathrsfs}
\usepackage[mathcal]{euscript}
\usepackage{color}
\usepackage{array}

\usepackage[pdftex]{hyperref}
\usepackage[parfill]{parskip}
\usepackage{url}

% math syntax
\newcommand{\nth}{n\ensuremath{^{\text{th}}} }
\newcommand{\ve}[1]{\ensuremath{\mathbf{#1}}}
\newcommand{\Macro}{\ensuremath{\Sigma}}
\newcommand{\rvec}{\ensuremath{\vec{r}}}
\newcommand{\omvec}{\ensuremath{\hat{\Omega}}}
\newcommand{\vOmega}{\ensuremath{\hat{\Omega}}}
%---------------------------------------------------------------------------
%---------------------------------------------------------------------------
\begin{document}
\begin{center}
{\bf NE 155/255, Fall 2019 \\
Solution Context and Tools\\
September 23, 2019}
\end{center}

\setlength{\unitlength}{1in}
\begin{picture}(6,.1) 
\put(0,0) {\line(1,0){6.25}}         
\end{picture}

We've started looking a little bit at how to play around with the transport 
equation. To venture much farther in thinking about solving the transport 
equation, it helps to remember what we're applying it to and some other bits 
of information that we use. 

As mentioned, we'll mostly focus on reactors in this class. A nuclear reactor 
is a three-dimensional structure consisting of complicated geometrical shapes 
made of variety of materials.	

\begin{compactitem} 
\item A unit cell usually consists of a fuel rod, gap, cladding and
      corresponding moderator. It is usually surrounded by similar cells. A
      fuel rod consists of fuel pellets.	
\item A fuel assembly usually consists of several hundred fuel rods (fuel 
      cells).	
\item A reactor core consists of several hundred fuel assemblies.	
\item Fuel assemblies and fuel rods are usually arranged in square or
      hexagonal lattice.	
\item Instead of fuel pellets, the fuel could be in the form of coated fuel
      particles (TRISO particles).
\item Coated fuel particles could be arranged into fuel compacts or pebbles
      (pebble bed reactor).	
\end{compactitem}

% \begin{figure}[h!]
    \begin{center}
    \includegraphics[keepaspectratio, width=\textwidth]{pwr-assembly.png}
    \end{center}
% \end{figure}
% PWR assembly diagram
% https://doi.org/10.1016/j.pnucene.2018.03.015

% \begin{figure}[h!]
    \begin{center}
    \includegraphics[keepaspectratio, width=\textwidth]{triso.png}
    \end{center}
% \end{figure}
% triso particle
% https://inldigitallibrary.inl.gov/sites/sti/sti/4336175.pdf

% \begin{figure}[h!]
    \begin{center}
    \includegraphics[keepaspectratio, width=\textwidth]{assembly-types}
    \end{center}
% \end{figure}
% assembly types diagram
% figure 10.4 from D&H

% \begin{figure}[h!]
    \begin{center}
    \includegraphics[keepaspectratio, width=\textwidth]{unit-cells}
    \end{center}
% \end{figure}
% unit cells
% fig 10.5 from D&H

\subsubsection*{PWR Specs (from Appendix H of Nuclear Reactor Theory)}

\begin{itemize}
\item{$\sim$200 fuel assemblies}
\item{Each assembly contains array of 17$\times$17 fuel rods}
\item{3.5 - 4 m tall, $\sim$3.5 m active diameter}
\item{UO$_2$ fuel, H$_2$O moderator and coolant, Zircaloy clad}
\item{$\sim$3\% enrichment}
\end{itemize}

\subsubsection*{BWR Specs (from Appendix H of Nuclear Reactor Theory)}

\begin{itemize}
\item{$\sim$700 fuel assemblies}
\item{Each assembly contains array of 8$\times$8 fuel rods}
\item{$\sim$3.5 m tall, $\sim$3.5 m active diameter}
\item{UO$_2$ fuel, H$_2$O moderator and coolant, Zircaloy clad}
\item{$\sim$2.5\% enrichment}
\end{itemize}

% \begin{figure}[h!]
    \begin{center}
    \includegraphics[keepaspectratio, width=\textwidth]{homog.png}
    \end{center}
% \end{figure}
% homogenization
% fig 10.6 from D&H

%-------------------------------------------------------------
%-------------------------------------------------------------
\pagebreak
% \vspace{-1 em}
\subsection*{Solution Context and Data}%
\begin{figure}[h!]
    \begin{center}
    \includegraphics[keepaspectratio, width = 4.5 in]{solver-map}
    \end{center}
    %\caption{Legendre polynomials}
    \label{fig:context}
\end{figure}

We have many different types of geometries and physics going on with the 
systems we're interested in. However, we take the same fundamental approach no 
matter what. Each component is incredibly important, but let's take a moment 
to talk about \textbf{nuclear data}.

We need a description of all of the physical interactions happening inside a 
nuclear reactor that we can use in our equation. An \textit{evaluated nuclear 
data file} is a collection of various data enabling to reconstruct, for each 
isotope, its cross-section's 

\begin{compactitem}
\item general information
\item resonance parameters 
\item angular distribution for emitted particles 
\item energy distribution for emitted particles 
\item energy-angle distribution for emitted particles 
\item thermal neutron scattering law data 
\item radioactivity and fission-product yield data 
\item multiplicities for radioactive nuclide production 
\item cross-sections for radioactive nuclide production
\end{compactitem}

There are many evaluations coming from various countries such as: USA, Europe, 
Japan, Russia, China, \dots Getting from experimental data (what we have of 
it) + theory (however accurate that is) to data that can actually be used in a 
code is no small feat. And, that process is sort of a mess...
\pagebreak

\begin{figure}[h!]
    \begin{center}
    \includegraphics[keepaspectratio, width = \textwidth]{data-flow}
    \end{center}
\end{figure}
\pagebreak

% https://www.epj-conferences.org/articles/epjconf/pdf/2017/22/epjconf_icrs2017_07003.pdf
The NJOY manual is approximately 800 pages.

We won't go through all of this, but it's important to have context 
about what this is and how confusing it can be. There is a lot of data and 
many formats. In computation you will pretty much only need to interface with 
ENDF and its equivalents. 

The data that we use is complicated, and can be quite different depending on 
the application we're interested in. Let's look at some. All plots shown here
were generated with the NNDC ENDF project at
\url{https://www.nndc.bnl.gov/exfor/endf00.jsp}


\begin{center}
\includegraphics[keepaspectratio, width = \textwidth]{u235xs}
\end{center}

\begin{center}
\includegraphics[keepaspectratio, width = \textwidth]{u238xs}
\end{center}

\begin{center}
\includegraphics[keepaspectratio, width = \textwidth]{hdxs}
\end{center}


%-------------------------------------------------------------
%-------------------------------------------------------------
\pagebreak
\subsection*{Physics Impacts}

We are often able to use knowledge about the physics to inform method 
development or, at the very least, choose which methods are more appropriate 
given our physics. 

For example, you may notice is how, in particular, fast and thermal reactor 
physics differ. We often need different codes to deal with LWRs and FRs. 

Many of the assumptions employed in traditional LWR methods do not apply:

\begin{compactitem}
\item Lack of a 1/E energy spectrum as a basis for the calculation of 
      resonance absorption.
\item Upscattering resulting from the thermal motion of the scattering nuclei 
      may be neglected.
\item Inelastic, (n, 2n), and anisotropic scattering are quite important.
\item Long mean free paths imply global coupling. That is, local reactivity 
      effects impact the entire core.  
\item The energy range where neutrons induce fission and the energy range 
      where the fission neutrons appear strongly overlap.
\end{compactitem}

Other physics considerations have high priority in FR methods:

\begin{compactitem}
\item Detailed energy modeling for resonance structure (core/reflector).
\item Transport and anisotropy effects are more important at high energy.
\end{compactitem}

In general, a distinct set of physics analysis and core design tools with 
tailored assumptions have been and are being developed for FR analysis.

\end{document}
